\documentclass[11pt,letterpaper]{article}
\usepackage{geometry}
 \geometry{
 letterpaper,
 total={170mm,257mm},
 left=20mm,
 top=20mm,
 }
\usepackage[utf8]{inputenc}
\usepackage[T1]{fontenc}
\usepackage{graphicx}
\usepackage[export]{adjustbox}
\graphicspath{ {images/} }
\usepackage{multicol}
\usepackage{nopageno}
\usepackage[sfdefault,light]{FiraSans}
\usepackage{titlesec}
\usepackage[english]{babel}

\hyphenchar\font=-1

\titleformat{\section}[runin]
  {\normalfont\small\bfseries}{\thesection}{.5em}{}

\setlength{\parindent}{0em}
\setlength{\parskip}{0em}
\renewcommand{\baselinestretch}{1.0625}


\title{Stanford Libraries AI Guiding Values}


\begin{document}


\begin{multicols}{2}

{\textbf{\huge Stanford Libraries \\ AI Guiding Values}}

\columnbreak

{\textbf{\small These guiding values bring the ethos of the library into our use of AI. These are intended for regular ongoing evaluation by everyone working on AI projects or steering AI programs at Stanford Libraries.}}
\end{multicols}


\section*{Ethical considerations}
Projects should explicitly consider the humanity, privacy, and safety of any subjects included in the data, and be transparent about decisions with ethical implications. Projects should ensure that the levels of access provided support the privacy of individuals represented.  Attention must also be given to the potential for the data to be used by third parties in combination with other sources to de-anonymize or de-pseudonymize individuals.Ethics review should be a part of the life of the project and a designated role.
\section*{Training for staff}
Since these techniques are not yet a core competence for Stanford Libraries (or libraries, generally), training opportunities are necessary to the success of the program and ought to be supported across the library for anyone who would like to be involved whether as engineer, project manager, domain expert, or data scientist. Training for those involved in these projects will include attention to the ethical implications of mining data at scale and making it publicly available. Projects themselves create opportunities for fostering a culture where everyone involved is a learner and at the same time carries a responsibility for others to learn. 
\section*{Need driven, not technology driven}
We do not need to look far to uncover needs, our subject specialists, archivists, metadata librarians, and others have a backlog of requests. But projects often require a range of tasks, of which only some can be addressed with AI technologies. AI should serve a critical need identified by the domain expert(s). Projects are driven by the mission to support research, preserve information, and enhance the work of library staff within and beyond Stanford.
\section*{Evaluation, re-evaluation, and continuous adjustment}
The nature of data-driven projects that influence production is that as the data changes, updates to methodology, risk assessment, ethical assessment, and outcomes are necessary. A virtuous cycle of review and renewal is part of the new way of working that needs to be built into this process beyond project development and approval. 
\section*{Diversity in representation}
 (a) Actively seek diversity of people and perspectives involved in the steering group and project teams by rotating participation, outreach to units not represented, and annual review of the distribution of participation. (b) Actively seek diversity of content and collections affected. We often fall back on convenience when applying technology; we take the path of least resistance. Libraries are prized as a public good in large part because our holdings are richly diverse and rare. (c) Actively seek diversity of audience served. 
Community Engagement: We engage with AI developments and work to elevate them not just at Stanford but in the wider community. 
\section*{Preservation and documentation}
Solving a particular ML problem can have wider re-use. SUL needs to provide for appropriate preservation of the data, transformed data, trained models, and any related documentation. Project teams need to approach the projects as a teaching and learning opportunity.

\begin{flushright}
Stanford Libraries AI Steering Group

February 24, 2021

\includegraphics[width=0.25\paperwidth, right]{StanfordLibraries-logo-cmyk}
\end{flushright}

\end{document}
